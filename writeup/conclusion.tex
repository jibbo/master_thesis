\chapter{Future Work}
\label{ch:futurework}
Given that the focus of this thesis in providing a novel and compiler-based tool to improve security, its state is prototypical and far from being optimal. However, it provides a solid base for further development in the area, which can start from what is here presented.

\section{Support for Additional Platforms}
Currently, \emph{DevArtist} is available only on arm processors for Android 7 Nougat, but it can be easily shipped for any Android platform, where \emph{Artist} is available, by re-compiling dex2oat within all the other Android versions and publish the packages on the Google Play Store. This would enable more people to benefit from this work.

\section{Optimizations}
The current algorithm which checks for SQL Injections is using a static vector for comparison. This issue should be addressed to substantially improve the accuracy of detection and reliability of the patching. The goal can be achieved by implementing a full-fledged SQL parser in Java which checks and enforce, at runtime, the constraints of the SQL query. Moreover, the current algorithm which detects the non hard-coded parts of the query string, at the compiler-level, could be optimized to increase its speed and lower its memory footprint.

\section{Discarding Root as Requirement}
\label{sc:discardroot}
Root is mandatory due to the fact that the on-device OAT file produced by the original application has to be replaced with the one compiled within our system and its location is protected by Android. However, a solution to this problem is provided in \cite{boxify}, which enables a light-weight but full-fledged sandboxing system without relying on OS modifications or root. In particular, it re-implements parts of the Android middleware in the sandbox and virtualize parts of the privileged filesystem, thus enabling apps to be installed without being root. Alternatively, as discussed in \ref{sc:deploy}, a custom ROM could replace the original dex2oat with our one and, by being a system app with higher privileges, it does not require root permission to install app. Projects such as CopperOS and LineageOS, could advertise this move as a way to improve the overall security, whilst \emph{DevArtist} would benefit by gaining more eyes (and hopefully improvement) on its source-code.

\section{Extensions}
One helpful extension of \emph{DevArtist} could be the possibility to choose between which of the three patches a user wants to apply on a per-app basis. This would greatly improve the usability for experimenters that would like to exclude risky patches and only use the safe ones.\newline
In addition, a system as \emph{DevArtist} can be used as a base to further improve security. In fact, by extending the auxiliary library it would be interesting to check at runtime if strings contained in variables are passwords and check if they are stored unencrypted. If that is the case, the project could provide "transparent" encryption for passwords stored in SharedPreferences, Files, Databases and so on, without requiring any modification from a developer.\newline
However, one of the best extensions the system might support is the ability to define a sort of \enquote{policy} for each vulnerability, by declaring the signature of a vulnerable method, and including in it also the way to patch it. This approach would offer to developers of companies to create custom policies for their own code and it would enable to speed up the code reviewing process.\newline
Moreover, this could even be extended more and create something similar to PatchDroid\cite{patchdroid}: given a repository for every vulnerability and platform, \emph{DevArtistGUI} could send to it information about a target app and the device on which it is running and receive back the proper version of \emph{dex2oat} and \emph{Codelib} to patch the vulnerabilities exposed by the selected app.\newline


\chapter{Conclusion}
The first contribution of this work used the empirical data in chapter \ref{ch:investigation} to show that developers still code insecurely, finding that unsafe randomness, obsolete hashing functions and insecure methods to query databases are still widely used on the top apps of Android's Play Store. This thesis also proposed a few threat models for the before-mentioned issues, to help understanding the necessity of the second contribution: a patching prototype described in chapter \ref{ch:devartist}. This solution's novelty manifests in the usage of the new Android on-device compiler and it also shows its advantages for the end user in respect to already existing tools. Despite the prototypical stage of the proposed system, it manages to achieve two of the main characteristics of an ideal tool and the automated tests, executed in chapter \ref{ch:evaluation}, suggest the usefulness of \emph{DevArtist} in the fight against malware. In fact, the evaluation's result are promising and hint that for the 49\% of the analyzed apps, the system reduces their attack surface while keeping them functional. Moreover, in contrast to expectation, these results also showed that changing the hashing algorithm in applications is safer than replacing the underlaying randomness, making the latter more of a risky patch than the former. Although \emph{Monkey Troop} enabled to run tests on a larger amount of apps than manually possible, which incremented the scale of the investigation, it also limited the precision of the results and other further findings, showing that there is a need for more research and development in the area. In conclusion, this thesis evaluated the need of patching vulnerabilities left open by developers, suggests a partial solution by creating a prototype addressing SQL Injections, unsafe Random usage and obsolete Hash Functions, thus showing, through evaluation, that patching those vulnerabilities can be done effectively at a compiler-level and increasing the overall security of a device.

%However, the success rate and other findings could have been more precise if \emph{Monkey Troop}, even tough it enabled to run tests on a larger amount of apps than manually possible, was capable to reach every execution path, thus requiring further development. In conclusion, final users who have root permissions can already harden the security of their devices by installing DevArtistGUI or wait for a Custom ROM which ships with it.






     